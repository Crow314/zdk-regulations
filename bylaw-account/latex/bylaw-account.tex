% arara: uplatex until !found('log', 'undefined references')
% arara: dvipdfmx


\documentclass[dvipdfmx, uplatex, 12pt]{jsarticle}

\usepackage{comment}
\usepackage{graphicx}
\usepackage{url}
\usepackage{booktabs}
\usepackage{multirow}
\usepackage{amsmath, amssymb}
\usepackage{spverbatim}

\usepackage[top=20truemm,bottom=25truemm,textwidth=42zw]{geometry}

\begin{document}

\begin{center}
    \begin{Large}
        全学学類・専門学群・総合学域群代表者会議 \\
        会計細則
    \end{Large}
\end{center}

\begin{flushright}
    令和5年2月22日

    全学学類・専門学群・総合学域群代表者会議 \; 決議
\end{flushright}

\vskip 2\baselineskip

\noindent
(目的)

第1条 \; この細則は、「全学学類・専門学群・総合学域群代表者会議について」の委任に基づく会計に関する事項を定めることを目的とする。

\vskip\baselineskip

\noindent
(会計責任者)

第2条 \; 全代会に会計責任者1名を置く。

2 \;
会計責任者は、総務委員長がこれを務める。

3 \;
会計責任者に事故あるときは、議長がこの任務を代行する。

\vskip\baselineskip

\noindent
(予算の作成)

第3条 \; 予算案の作成は、会計責任者がこれを行う。

2 \;
予算案は、会計責任者が議長にこれを提出する。

\vskip\baselineskip

\noindent
(繰越金)

第4条 \; 会計年度の終了までに執行されなかった剰余金は、次年度予算において繰越金として収入計上する。

\vskip\baselineskip

\noindent
(予備費)

第5条 \; 予期し難い予算の不足に備えるため、予算に予備費を計上しなければならない。

2 \;
予備費は、この予算額を予算総額の10分の1以上としなければならない。
ただし、予算総額の10分の1が繰越金を除く収入予算額を超過する場合は、予備費の予算額は、繰越金を除く収入予算額以上とすれば良い。

3 \;
予備費の支出は、委員長連絡会の決議による。

\vskip\baselineskip

\noindent
(決算)

第6条 \; 決算は、会計責任者がこれを行う。

2 \;
決算は、会計年度の終了から120日以内にこれを報告しなければならない。

\vskip\baselineskip

\noindent
(予算の執行)

第7条 \; 金銭の支出のあるときは、これを会計責任者に届出なければならない。

2 \;
支出の届出にあたっては、その領収書又は請求書を会計責任者に提出しなければならない。

3 \;
支出の届出は、その領収書又は請求書の発行日から起算して30日以内に、これを行わなければならない。

4 \;
支出の届出は、原則として、領収書又は請求書の発行日と異なる会計年度に、これを行うことはできない。

5 \;
会計責任者は、支出の届出があったとき、速やかに予算を執行しなければならない。
ただし、支出の届出に不備があったとき又はその予算において認められない支出があったときはこの限りではない。

\vskip\baselineskip

\noindent
(特別会計)

第8条 \; 特別会計は、以下に掲げる何れかに該当し、全代会一般の会計(以下「一般会計」という。)と区分する必要のあるときに設けることができる。

\begin{enumerate}
    \item 特定の事業に関する会計
    \item その他一般会計と区分する必要のあるもの
\end{enumerate}

2 \;
特別会計は、これの予算案の承認を以て設置される。

3 \;
設置期間が1ヶ年未満の特別会計は、設置期間が単一の会計年度に含まれるかに拘わらず、単年度の会計としてこれを扱う。

4 \;
設置期間が終了した特別会計は、期末を迎えたものとし、決算を行う。

5 \;
一般会計は、期末を迎えたとき、その時点における特別会計の財務状況を決算に反映しなければならない。

\vskip\baselineskip

\noindent
(特別会計の会計責任者)

第9条 \; 特別会計は、これに専任の会計責任者を置く。
特別会計の会計責任者についての規程は、一般会計の会計責任者にかかる規程を準用する。

2 \;
第2条第2項の規定に拘わらず、特別会計の会計責任者を総務委員長以外が務めることは、これを妨げない。

3\;
特別会計の会計責任者を一般会計の会計責任者が兼任することは、これを妨げない。

\vskip\baselineskip

\noindent
(特別会計の繰入金)

第10条 \; 特別会計は、一般会計から資産及び負債を、これに繰り入れることができる。

2 \;
特別会計は、設置期間が終了したとき、これの資産及び負債を一般会計に繰り出さなければならない。

\vskip\baselineskip

\noindent
(翌年度の事業に関する特別会計)

第11条 \; 全代会は、翌年度に実施する事業について、特別会計の予算を編成することができる。

2 \;
前項の規定により編成された特別会計の予算は、事業の実施年度における本会議決議により改めることができる。

\vskip\baselineskip

\noindent
(会計責任者の業務の委任)

第12条 \; 会計責任者は、その任務に係る業務の一部を、総務委員その他の全代会構成員又は準構成員に委任することができる。

\section*{附則}

本細則は、令和5年2月22日から施行する。

\end{document}
