% arara: uplatex until !found('log', 'undefined references')
% arara: dvipdfmx


\documentclass[dvipdfmx, uplatex, 12pt]{jsarticle}

\usepackage{comment}
\usepackage{graphicx}
\usepackage{url}
\usepackage{booktabs}
\usepackage{multirow}
\usepackage{amsmath, amssymb}
\usepackage{spverbatim}

\usepackage[top=20truemm,bottom=25truemm,textwidth=42zw]{geometry}

\renewcommand{\labelenumi}{(\arabic{enumi})}

\begin{document}

\begin{center}
    \begin{Large}
        全学学類・専門学群・総合学域群代表者会議について
    \end{Large}
\end{center}

\begin{flushright}
    令和2年1月29日

    全学学類・専門学群・総合学域群代表者会議 \; 決定

    改正 令和5年3月1日
\end{flushright}

\vskip 2\baselineskip

\subsection*{前文}

本決定は、「筑波大学における学生の組織及びクラス連絡会等について」(以下「副学長決定」という。)第84項に基づき、全学学類・専門学群・総合学域群代表者会議(以下「全代会」という。)の円滑な運営を図るために運営の細目について定めるものである。

\vskip\baselineskip

\subsection*{第一章 組織}

第一条 \;
全代会は、「筑波大学の学生組織等について」(以下「学長決定」という。)第20項に基づき、全学に渡る学生生活および教育に関する事項等について討議し、意見等をまとめることを目的とする。

2 \;
会計責任者は、総務委員長がこれを務める。

\vskip\baselineskip

第二条 \;
全代会の構成員は、「学長決定」第2項(3)および「副学長決定」第4項に基づき、各学類、体育専門学群、芸術専門学群および総合学域群(以下「学類等」という。)のクラス代表者会議からそれぞれ選出された座長及び副座長とする。

2 \;
前項によって規定される全代会の構成員を「座長団」と呼称する。

\vskip\baselineskip

第三条 \;
全代会は、「副学長決定」第5項に基づき、その統括を目的として議長および副議長を置く。

2 \;
全代会の代表者は、議長をもって充てる。

3 \;
議長および副議長を「議長団」と呼称する。

\vskip\baselineskip

第四条 \;
議長団は、「学長決定」第25項に基づき、全代会でまとめた意見等を、学生担当副学長に報告するとともに、全学の学生を代表して意見を述べる。

\vskip\baselineskip

第五条 \;
議長団の選出方法は、「学長決定」第23項および「副学長決定」第32項から第34項に従う。

2 \;
議長および副議長の立候補者が定員以内であった場合は信任投票を行い、構成員の過半数の承認をもってこれを定める。

\vskip\baselineskip

第六条 \;
議長団の解任方法は、「副学長決定」第35項から第38項に従う。

2 \;
議長および副議長は、「副学長決定」第37項に規定されている内容のほかに、除籍、移籍、転出となるなど、会議への出席が不能になった場合にも離任する。

3 \;
議長および副議長が離任した場合は、「副学長決定」第39項に従い、10日以内に後任の選挙を行う。

\vskip\baselineskip

第七条 \;
議長団の任期は、「副学長決定」第54項に基づき、選出日より翌年度の授業開始日までとする。
また、後任の任期は、同項に従い、前任者の残任期間とする。

\vskip\baselineskip

第八条 \;
新年度の全代会が召集されるまでの当会議に関する諸事務は、「副学長決定」第55項に従い、前年度の構成員がこれを代行する。

2 \;
「副学長決定」第34項に規定される選挙の運営は、前年度の構成員が行う。

\vskip\baselineskip

第九条 \;
全代会は審議の能率化をはかるため、「副学長決定」第74項に基づき、常任委員会を置く。

2 \;
座長団は、原則としていずれかの常任委員会のうち一つ以上に所属する。
ただし、議長団が常任委員会に所属する必要はない。

\vskip\baselineskip

第十条 \;
全代会は常任委員会の所管に属さない問題または臨時で取り組む必要のある問題を扱う場合、「副学長決定」第80項に基づき、特別委員会を設置する。

\subsection*{第二章 会議}

第十一条 \;
全代会の進行は、「副学長決定」第58項に基づき、議長がつかさどる。

\vskip\baselineskip

第十二条 \;
会議は、「学長決定」第21項および第22項に基づき開催される。

2 \;
前項における会議を「本会議」と呼称する。

\vskip\baselineskip

第十三条 \;

座長団が授業等によりやむを得ず会議に出席できない場合に限り、当該学類等の学生は、当該座長団を代理して会議に出席することができる。

2 \;
前項に定める学生を、「座長代理」あるいは「副座長代理」と呼称する。

3 \;
座長団が持つ権利は、当該会議に限りすべて座長代理および副座長代理に委任される。

4 \;
座長代理および副座長代理は、代理される座長団の所属する学類の中から、当該クラス代表者会議議長が指名したものがこの任にあたる。

5 \;
代理される座長団は、会議開催の2日前までに、欠席の理由、代理出席者の氏名およびクラス代表者会議議長の署名が明記された代理出席書を、全代会の議長に提出し、受理されなければならない。

\vskip\baselineskip

第十四条 \;
議長は、本会議に先立ち議案に対する意見を聴取する場として「意見聴取会」を開催することができる。

2 \;
意見聴取会は、「副学長決定」第71項に基づき、同第58項から第70項の規定に準じて進行する。

第十五条 \;
会議の開催手続きは、「副学長決定」第60項から第63項に従う。

\vskip\baselineskip

第十六条 \;
議案が発議されてから 30 分以上遅刻した者は、その議案に対する議決権を失う。

\vskip\baselineskip

第十七条 \;
全代会の意見は、「学長決定」第23項および第24項に基づき構成員の過半数をもってまとめる。

2 \;
本会議で発議された議案において承認投票数が全代会構成員の過半数を超えた場合、その議案について全代会の承認があったものとする。

3 \;
本会議で発議された議案において否認投票数が全代会構成員の過半数を超えた場合、その議案は以降の本会議で発議できないものとする。

4 \;
本会議で発議された議案において保留投票数が全代会構成員の過半数を超える、もしくは承認・保留・否認いずれの項目においても投票数が全代会構成員の過半数に満たない場合、その議案は保留とする。
保留された議案は、以降の本会議において再発議することができる。

\vskip\baselineskip

第十八条 \;
本会議における議案の発議者は、採決において保留または否認に投票をした全代会構成員に対しその理由を問うことができる。
問われた対象者は保留または否認に投じた理由を述べなければならない。

第十九条 \;
構成員は、会議中に動議議案を発議することができる。

2 \;
動議議案の発議は、各学類等の座長が行う。

3 \;
動議議案は、構成員の過半数の賛成により認可される。

4 \;
認可された動議議案は、「副学長決定」第60項に規定される議案と同様に審議し、その決議の結果は参考意見として議長団によって学生担当副学長に提出される。

\vskip\baselineskip

第二十条 \;
構成員は、学生担当副学長あるいは他の構成員に対し、報告文書を提出することができる。

\vskip\baselineskip

第二十一条 \;
会議の議事録は、「副学長決定」第64項に基づき、これを作成・保管し、随時公開する。

2 \;
議事録は、作成後に議長がこれを確認し、承認した後ただちに公開する。

\vskip\baselineskip

第二十二条 \;
全代会は、「副学長決定」第67項に基づき、参考人を招聘することができる。

2 \;
参考人は、構成員の意見を聞いて、議長がこれを招聘する。

\subsection*{第三章 委員会および委員長連絡会}

第二十三条 \;
常任委員会は、「副学長決定」第74項および第75項に基づき設置され、第75項および第76項に規定された業務を行う。

2 \;
総務委員会は、次に掲げる部門および、各部門の業務を統括する部門長を置く。

\begin{enumerate}
    \item 事務部門...... 議事進行の補佐及び全代会運営に係る庶務
    \item 情報部門...... 全代会の情報環境整備・サービス提供による情報流通活性
\end{enumerate}

3 \;
広報委員会は、次に掲げる部および、各部の業務を統括する部長を置く。

\begin{enumerate}
    \item 編集部...... 全代会広報紙の執筆・編集
    \item 制作部...... 全代会広報活動に必要な広報物の制作
\end{enumerate}

\vskip\baselineskip

第二十四条 \;
常任委員会は、「副学長決定」第77項に基づき、全代会構成員および必要に応じ当該クラス代表者会議が推薦し、全代会の議長が任命した者によって構成される。

2 \;
常任委員の任命および罷免は、「副学長決定」第78項に基づき、全代会の議長が全代会の意見を聞いてこれを行う。

3 \;
すべての常任委員会は、1名以上の座長団を含む。

4 \;
「副学長決定」第77項における「必要に応じ全代会の構成員が推薦し、全代会の議長が任命した者」を「専門委員」と呼称し、全代会の準構成員とする。

\vskip\baselineskip

第二十五条 \;
各常任委員会の委員長(以下「常任委員長」という。)は、「副学長決定」第76項に基づき、当該委員会に所属する座長団の中から、その委員会の構成員の互選によって選出する。

2 \;
常任委員会の委員長と議長との兼任は、原則としてこれを認めない。

3 \;
複数委員会の委員長を兼任することは、これを妨げない。

4 \;
総務委員長と第二十条の 2 に示した各部門長との兼任、ならびに広報委員長と第二十条の3に示した各部長との兼任は、これを妨げない。

\vskip\baselineskip

第二十六条 \;
常任委員長の選出は、次に定める方法で行う。

\begin{enumerate}
    \item 全表決者の過半数の得票者を委員長とする。過半数の得票者がないときは、上位2名の決選投票による
    \item 不在者投票および代理投票は認めない。
    \item 立候補者が1名の場合は、信任投票を行う。
\end{enumerate}

2 \;
各常任委員会の選挙には、立会人として議長もしくは副議長が同席する。

\vskip\baselineskip

第二十七条 \;
常任委員長の解任方法は、次に定める方法で行う。

\begin{enumerate}
    \item 委員長が当該委員会において辞意を表明し、委員会構成員の過半数の承認が得られた場合、委員長は辞任する。
    \item 当該委員会の構成員の4分の1以上によって、委員長の解任の請求が全代会の議長に提出された場合、委員会での投票に付し、委員会構成員の過半数の解任支持票があるとき、委員長は解任される。
    \item 委員長の罷免は、全代会の議長が当該委員会の意見を聞いてこれを行うことができる。
    \item 委員長が退学もしくは停学・休学・卒業・除籍・移籍・転出等で委員会への出席が不能となった場合、委員長は離任する。
    \item 委員長が当該委員を罷免された場合、委員長は離任する。
    \item 委員長が座長団を辞任または離任し、あるいは解任された場合、委員長は原則として離任する。
    \item 委員長が議長に選出された場合、委員長は離任する。
    \item 委員長が、辞任または離任し、あるいは解任された場合、14日以内に後任委員長の選挙を行う。
\end{enumerate}

\vskip\baselineskip

第二十八条 \;
常任委員長の任期は、選出された年度の4月1日から翌年の3月30日までとする。ただし、前条に規定された方法で解任された委員長の後任者の任期は、前任者の残任期間とする。

2 \;
委員長は、前項に規定される任期中における授業開始日から座長団選出が行われる期間までの間は、前条(6)の規定にかかわらず離任しない。
ただし、委員長が座長団選挙の結果座長団たる資格を失った場合、委員長は離任する。

\vskip\baselineskip

第二十九条 \;
特別委員会は、「副学長決定」第80項から第83項に基づき設置する。

\vskip\baselineskip

第三十条 \;
特別委員会の構成は、本決定第十八条の内容に準じる。

2 \;
前項の規定にかかわらず、特別委員会は座長団を構成員に含む必要はない。

第三十一条 \;
特別委員会の委員長(以下、「特別委員長」という。)は、「副学長決定」第78項に基づき、その委員会の構成員の中から互選によって選出する。

2 \;
特別委員長と議長団との兼任は、これを妨げない。

3 \;
特別委員長と常任委員長あるいはほかの特別委員会の委員長との兼任は、これを妨げない。

\vskip\baselineskip

第三十二条 \;
特別委員長の選出は、本決定第二十三条に定める方法と同様に行う。

\vskip\baselineskip

第三十三条 \;
特別委員長の解任方法は、本決定第二十四条に準じる。

2 \;
議長が特別委員会の委員長を兼任している場合、解任請求の提出先および罷免の判断は本決定で別に定める委員長連絡会が代行する。

\vskip\baselineskip

第三十四条 \;
各委員会は、「学長決定」および「副学長決定」並びに本決定の定めるところに抵触しない限りにおいて細則を定めることができる。

\vskip\baselineskip

第三十五条 \;
「副学長決定」第59項(3)「その他の各委員長」における議長の代行は次の順位による。

\begin{enumerate}
    \item 学内行事委員長
    \item 教育環境委員長
    \item 生活環境委員長
    \item 調査委員長
    \item 広報委員長
    \item 座長団たる特別委員長
\end{enumerate}

2 \;
前項(6)における特別委員長が複数名いる場合は、特別委員長に選出された日時が早い順に上位とする。

\vskip\baselineskip

第三十六条 \;
全代会の方針決定の場として、委員長連絡会を設置する。

\vskip\baselineskip

第三十七条 \;
委員長連絡会は、議長団、常任委員長および特別委員長で構成される。

2 \;
議長団は委員長連絡会を統括し、議事の進行をつかさどる。

3 \;
前項の規定にかかわらず、本決定第三十条第2項に規定される議長たる特別委員長の解任請求および罷免に関する議題を扱うときは、議長は議事の進行を行わない。

4 \;
委員長連絡会は、必要に応じて参考人を招聘できる。

第三十八条 \;
委員長連絡会にて議決された内容は、議長が会議にて座長団に報告する。

\subsection*{第四章 会計}

\noindent
(財産)

第三十八条の2 \;
全代会は、金銭、物品その他の財産を所有する。

\vskip\baselineskip

\noindent
(会計年度)

第三十八条の3 \;
全代会の会計年度は、毎年4月1日に始まり翌年3月末日に終わるものとする。

\vskip\baselineskip

\noindent
(予算)

第三十八条の4 \;
全代会は、毎年度、その年度の予算を編成しなければならない。

2 \;
全代会の予算は、本会議の承認により成立する。

\vskip\baselineskip

\noindent
(決算)

第三十八条の5 \;
全代会の決算は、本会議の承認を受けなければならない。

\vskip\baselineskip

\noindent
(特別会計)

第三十八条の6 \;
全代会は、特定の目的のために特別会計を設けることができる。

\vskip\baselineskip

\noindent
(会計監査)

第三十八条の7 \;
会計監査は、第三十九条に定める監察役がこれを行う。

2 \;
監察役は、全代会の全ての会計経理を監査する権利及び義務を有する。

3 \;
監察役は、第三十八条の5に定める決算報告の際に、会計監査の結果を報告しなければならない。

\vskip\baselineskip

\noindent
(会計細則)

第三十八条の8 \;
会計に関する細則は、これを別に定める。

\subsection*{第五章 その他全代会の運営に関する事項}

第三十九条 \;
全代会は、「副学長決定」第40項から第42項に基づき、全代会の不信任案を受理し、公正な信任投票を実施するために、全代会と独立した機関として監察役を置く。

2 \;
監察役は、「副学長決定」第41項に基づき、全代会構成員の中から翌年度座長団を続ける意思のないことを表明したものの中から、適任と思われるもの2名を翌年度の監察役として選出する。

\subsection*{第六章 改廃}

第四十条 \;
本決定の改廃について全代会の構成員の5 分の1 以上から提案された場合、全代会はこれを審議する。

\vskip\baselineskip

第四十一条 \;
学長もしくは副学長から本決定の改廃を求められた場合、および議長が本決定を改廃すべきと認めた場合、全代会はこれを審議する。

\vskip\baselineskip

第四十二条 \;
改正の議決については、他の議案と同様に行う。

\vskip\baselineskip

第四十三条 \;
廃止についての議案は、全会一致のとき可決される。

\vskip\baselineskip

附記

本決定は、令和 2年 1月 29日から施行する。

\vskip\baselineskip

附記

本決定は、令和 3年 4月 1日から施行する。

\vskip\baselineskip

附記

本決定は、令和 4年 2月 27日から施行する。

\vskip\baselineskip

附則

本決定は、令和5年3月1日から施行する。

\end{document}
